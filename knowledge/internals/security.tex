\section{Security}

\subsection{Security system components}


\subsection{Object protection}
\begin{itemize}
    \item essence of discretionary access control and auditing
    \item Theoretically, anything managed by the executive object manager
    \item objects that are not exposed to user mode (such as driver objects) are usually not protected
    
\end{itemize}

To control who can manipulate an object, the security system must first be sure of each user’s identity.

When a process requests a handle to an object, the object manager and the security system use the  {\bf  caller’s security identification] and the {\bf object’s security descriptor} to determine whether the caller should be assigned a handle that grants the process access to the object it desires.

thread can assume a different security context than that of its process. This mechanism is called {\bf impersonation}.

It’s important to keep in mind that all the threads in a process share the same handle table, so when a
thread opens an object—even if it’s impersonating—all the threads of the process have access to the
object.

Sometimes, validating the identity of a user isn’t enough for the system to grant access to a resource that
should be accessible by the account. Windows achieves this kind of intra-user isolation with the {\bf Windows integrity mechanism}, which  implements integrity levels.

The Windows integrity mechanism is used by:
\begin{itemize}
    \item User Account Control (UAC)
    \item User Interface Privilege Isolation (UIPI)
    \item AppContainers
\end{itemize}




\subsubsection{Access checks}

thread must specify  up front, at the time that it opens an object, what types of actions it wants to perform on the object.

The object manager calls the SRM to perform access checks based on a thread’s desired access. If the access is granted, a handle is assigned to the thread’s process with which the thread (or other threads in the process) can perform further operations on the object.

Events causing the object manager to perform security access validation:
\begin{itemize}
    \item a thread opening an existin object using a name
    \item a process references an object using an existing handle
\end{itemize}

{\bf thread opening an existin object using a name}:
\begin{enumerate}
    \item Obj manager call \verb+ObpCreateHandle+ to create an entry in the process handle table that becomes associated with the object.
    \item \verb+ObpCreateHandle+ call \verb+ObpGrantAccess+ to see if the thread has permission to access the object
    \item \verb+ObpGrantAccess+ call \verb+ObCheckObjectAccess+ with the {\bf Security context} represented by an {\bf access token} the access type ({\bf access mask}) and a pointer to the object.
    \item \verb+ObCheckObjectAccess+ lock the object’s security descriptor and the security context of the thread (to prevent their modification)
    \item \verb+ObCheckObjectAccess+ then calls the object’s {\bf security method} to obtain the security settings of the object. 
    \item \verb+ObCheckObjectAccess+ then call \verb+SeAccessCheck+ from SRM with the object’s security information, the security identity of the thread and the access that the thread is requesting which return \verb+true+ or \verb+false+
    \item On success \verb+ObpCreateHandle+ calls the executive function \verb+ExCreateHandle+ to create the entry in the process handle table
\end{enumerate}


{\bf process references an object using an existing handle}:
The types of accesses the threads in the process are granted through the handle are stored with the handle by the object manager.

Some System services (for exemple NtWriteFile) uses the object manager function \verb+ObReferenceObjectByHandle+ to obtain a pointer to the file object from the handle


\verb+ObReferenceObjectByHandle+ accepts the access that the caller wants from the object as a parameter. After finding the handle entry in the process handle table, \verb+ObReferenceObjectByHandle+ compares the access being requested with the access granted for the thread.

The Windows security functions also enable Windows applications to define their own private objects and to call on the services of the SRM (through the \verb+AuthZ+ user-mode APIs) to enforce the Windows security model on those objects. Many kernel-mode functions that the object manager and other executive components use to protect their own objects are exported as Windows user-mode APIs. The user-mode equivalent of \verb+SeAccessCheck+ is the \verb+AuthZ API AccessCheck+. Windows applications can therefore leverage the flexibility of the security model and transparently integrate with the authentication and administrative interfaces that are present in Windows.

The essence of the SRM’s security model is an equation that takes three inputs: the security identity of a
thread, the access that the thread wants to an object, and the security settings of the object. The output is
either yes or no and indicates whether the security model grants the thread the access it desires. The
following sections describe the inputs in more detail and then document the model’s access-validation
algorithm.




\subsubsection{Access mask}
\href{https://learn.microsoft.com/en-us/windows/win32/secauthz/access-mask-format}{Access Mask Format}
\subsection{Tokens}

\subsubsection{Access token}
\subsubsection{Impersonation}
\subsubsection{Restricted tokens}
