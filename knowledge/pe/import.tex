\section{Import section (.idata)}
\href{https://learn.microsoft.com/en-us/windows/win32/debug/pe-format#the-idata-section}{The .idata} section specifies which symbols (functions and data) the binary imports from shared libraries.

When the loader resolves dependencies, it writes the resolved addresses into the {\bf Import Address Table (IAT)}.


\subsection{Import Descriptor Table}
The Import Descriptor Table is a Data Directory located at \verb+_IMAGE_NT_HEADERS.IMAGE_OPTIONAL_HEADER.IMAGE_DATA_DIRECTORY[IMAGE_DIRECTORY_ENTRY_IMPORT].VirtualAddress+

It consists of an array of \verb+IMAGE_IMPORT_DESCRIPTOR+ structures, each one of them is for a DLL.

It doesn’t have a fixed size, so the last \verb+IMAGE_IMPORT_DESCRIPTOR+ of the array is zeroed-out (NULL-Padded) to indicate the end of the Import Directory Table.

\begin{verbatim}
typedef struct _IMAGE_IMPORT_DESCRIPTOR {
    union {
/*0x00*/        DWORD   Characteristics;
/*0x00*/        DWORD OriginalFirstThunk;
    } DUMMYUNIONNAME;
/*0x04*/    DWORD   TimeDateStamp;
/*0x08*/    DWORD   ForwarderChain;
/*0x0c*/    DWORD   Name;
/*0x10*/    DWORD  FirstThunk;
} IMAGE_IMPORT_DESCRIPTOR;
typedef IMAGE_IMPORT_DESCRIPTOR UNALIGNED *PIMAGE_IMPORT_DESCRIPTOR;    
\end{verbatim}


\begin{itemize}
    \item \verb+OriginalFirstThunk+: RVA of the {\bf Import Lookup Table (ILT)} (an array of \verb+IMAGE_THUNK_DATA+). The name \verb+Characteristics+ is used in \verb+Winnt.h+, but no longer describes this field
    \item \verb+FirstThunk+: RVA of the {\bf Import Address Table (IAT)} (an array of \verb+IMAGE_THUNK_DATA+)
    \item \verb+TimeDateStamp+: A time date stamp, that’s initially set to 0 if not bound and set to -1 if bound. In case of an unbound import the time date stamp gets updated to the time date stamp of the DLL after the image is bound. In case of a bound import it stays set to -1 and the real time date stamp of the DLL can be found in the {\bf Bound Import Directory Table} in the corresponding \verb+IMAGE_BOUND_IMPORT_DESCRIPTOR+.
    We’ll discuss bound imports in the next section.
    \item \verb+ForwarderChain+: The index of the first forwarder chain reference. This is something responsible for DLL forwarding. (DLL forwarding is when a DLL forwards some of its exported functions to another DLL.)
    \item \verb+Name+: An RVA of an ASCII string that contains the name of the imported DLL
\end{itemize}


{\bf Bound imports}: essentially means that the import table contains fixed addresses for the imported functions. These addresses are calculated and written during compile time by the linker.

Using bound imports is a speed optimization, it reduces the time needed by the loader to resolve function addresses and fill the IAT, however if at run-time the bound addresses do not match the real ones then the loader will have to resolve these addresses again and fix the IAT.


{\bf ILT versus IAT}: We need to look a little at how the loader works. When loading the program, the loader will load the DLLs and replace the name of the DLL functions in the import table
with their address. Well, in fact, this is where it is done, the table pointed to by the \verb+Characteristics+ field contains and will always contain the name of the DLL functions while the one pointed to by \verb+FirstThunk+ will be changed by the Windows loader


\subsection{Import Lookup Table (ILT)}

An import lookup table is an array of 32-bit numbers for PE32 or an array of 64-bit numbers for PE32+.

Each entry uses the bit-field format that is described in the following table. In this format, bit 31 is the most significant bit for PE32 and bit 63 is the most significant bit for PE32+. The collection of these entries describes all imports from a given DLL. The last entry is set to zero (NULL) to indicate the end of the table.

The ILT is essentially a table of names or references, it tells the loader which functions are needed from the imported DLL.

The ILT consists of an array of 32-bit numbers (for PE32) or 64-bit numbers for (PE32+), the last one is zeroed-out to indicate the end of the ILT.

Each entry of these entries encodes information as follows:
\begin{itemize}
    \item Bit 31/63 (most significant bit): This is called the Ordinal/Name flag, it specifies whether to import the function by name (bit unset) or by ordinal (bit set).
    \begin{itemize}
        \item PE32: bitmask \verb+0x80000000+
        \item PE32+: bitmask \verb+0x8000000000000000+
    \end{itemize}
    \item Bits 15-0: if import by Ordinal these bits are used to hold the 16-bit ordinal number that will be used to import the function, bits 30-15/62-15 for PE32/PE32+ must be 0.
    \item Bits 30-0: if import by Name these bits are used to hold an RVA of a Hint/Name table. For PE32+ bits 62-31 must be zero. 
\end{itemize}

\begin{verbatim}
typedef struct _IMAGE_THUNK_DATA32 {
    union {
/*0x00*/        uint32_t* Function;             // address of imported function
/*0x00*/        uint32_t  Ordinal;              // ordinal value of function
/*0x00*/        PIMAGE_IMPORT_BY_NAME AddressOfData;        // RVA of imported name
/*0x00*/        DWORD ForwarderStringl              // RVA to forwarder string
    } u1;
} IMAGE_THUNK_DATA32, *PIMAGE_THUNK_DATA32;
\end{verbatim}

{\bf Ordinal Import}:
This is actually importing functions based on a number. Of course, the imported function must first call a function that is also exported ordinally! But the problem with this kind of export is that we are forced to always add the exported functions (we must always increment the export number of the function) and we cannot delete an exported function easily, otherwise we risk messing up all the rest of the numbers and the users who imported function x will actually import function x-1 for example, which is not very nice for the user of your DLL.

{\bf Name Import}:
As its name suggests, we will import the functions by their name, which is much more practical and is therefore much more widespread these days

\subsection{Hint/Name Table}

\begin{verbatim}
typedef struct _IMAGE_IMPORT_BY_NAME {
    WORD    Hint;
    CHAR   Name[1];
} IMAGE_IMPORT_BY_NAME, *PIMAGE_IMPORT_BY_NAME;    
\end{verbatim}

\begin{itemize}
    \item \verb+Hint+: An index into the export name pointer table. A match is attempted first with this value. If it fails, a binary search is performed on the DLL's export name pointer table. 
    \item \verb+Name+: A null-terminated string that contains the name of the function to import.
\end{itemize}

NOTE: The \verb+BYTE+ designated at Name of course only marks the beginning of the character array of the imported function name as the name can be larger than one character.

The first \verb+WORD+ is the hint, which you must be skipping over if you already have the name. 



\subsection{Import Address Table (IAT)}

The structure and content of the import address table are identical to those of the import lookup table, until the file is bound. During binding, the entries in the import address table are overwritten with the 32-bit (for PE32) or 64-bit (for PE32+) addresses of the symbols that are being imported. These addresses are the actual memory addresses of the symbols, although technically they are still called "virtual addresses." The loader typically processes the binding.