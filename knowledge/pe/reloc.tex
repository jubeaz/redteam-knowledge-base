\section{Relocation section (.reloc)}

The Base Relocation Table field in the optional header data directories gives the number of bytes in the base relocation table.

The base relocation table is divided into blocks. Each block represents the base relocations for a 4K page. Each block must start on a 32-bit boundary.

The loader is not required to process base relocations that are resolved by the linker, unless the load image cannot be loaded at the image base that is specified in the PE header.

Each base relocation block starts with the following structure:

\begin{verbatim}
typedef struct _IMAGE_BASE_RELOCATION {
/*0x00*/    DWORD   pageRVA;
/*0x04*/    DWORD   SizeOfBlock;
} IMAGE_BASE_RELOCATION;
typedef IMAGE_BASE_RELOCATION UNALIGNED * PIMAGE_BASE_RELOCATION;
\end{verbatim}

\begin{itemize}
    \item \verb+VirtualAddress+: Relative (to image base) Virtual Address location 
    \item verb+SizeOfBlock+: The total number of bytes in the base relocation block, including the Page RVA and Block Size fields and the Type/Offset fields that follow. 
\end{itemize}

After the block size, there is a list of 2 byte pairs denoting the patching instructions.

\begin{itemize}
    \item Stored in the high 4 bits of the WORD, a value that indicates the type of base relocation to be applied. For more information, see \href{https://learn.microsoft.com/en-us/windows/win32/debug/pe-format#base-relocation-types}{Base Relocation Types}.
    \item Stored in the remaining 12 bits of the WORD, an offset from the starting address that was specified in the Page RVA field for the block. This offset (\verb+ImageBase + pageRVA+) specifies where the base relocation is to be applied. 
\end{itemize}

\subsection{Relocating}
To apply a base relocation, the difference is calculated between the preferred base address and the base where the image is actually loaded. If the image is loaded at its preferred base, the difference is zero and thus the base relocations do not have to be applied.

Each relocation entry gets processed by adding the RVA of the page to the image base address, then by adding the offset specified in the relocation entry, an absolute address of the location that needs fixing can be obtained.

Below loop will fix up the required memory locations. It works by:
\begin{itemize}
    \item finding the relocation table and cycling through the relocation blocks
    \item getting the nuber of required relocations in each relocation block
    \item reading bytes in the specified relocation addresses
    \item applying delta (between source and destination imageBaseAddress) to the values specified in the relocation addresses
    \item writing the new values to specified relocation addresses
    \item repeating the above until the entire relocation table is traversed
\end{itemize}



\begin{itemize}
    \item \href{https://www.ired.team/offensive-security/code-injection-process-injection/process-hollowing-and-pe-image-relocations}{Process Hollowing and Portable Executable Relocations}
    \item \href{http://malwareid.in/unpack/unpacking-basics/pe-relocation-table}{What Every Malware Analyst Should Know About PE Relocations}
\end{itemize}